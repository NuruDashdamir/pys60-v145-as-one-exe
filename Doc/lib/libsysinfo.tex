% Copyright (c) 2005-2007 Nokia Corporation
%
% Licensed under the Apache License, Version 2.0 (the "License");
% you may not use this file except in compliance with the License.
% You may obtain a copy of the License at
%
%     http://www.apache.org/licenses/LICENSE-2.0
%
% Unless required by applicable law or agreed to in writing, software
% distributed under the License is distributed on an "AS IS" BASIS,
% WITHOUT WARRANTIES OR CONDITIONS OF ANY KIND, either express or implied.
% See the License for the specific language governing permissions and
% limitations under the License.

\section{\module{sysinfo} ---
    Access to system information}
\label{sec:sysinfo}
\declaremodule{extension}{sysinfo}
\platform{S60}
\modulesynopsis{Package supporting system information fetching in S60 devices.}
The \textsf{sysinfo} module offers an API for checking the system 
information of a S60 mobile device. 

\begin{notice}[note]
The method \code{ring{\_}type} is not available for S60 1st Edition.
\end{notice}

The \module{sysinfo} module has the following functions:

\begin{funcdesc}{active_profile}{}
Returns the current active profile as a string, which can be one of the 
following: \code{'general'}, \code{'silent'}, \code{'meeting'}, 
\code{ 'outdoor'}, \code{'pager'},\code{ 'offline'}, ,\code{ 'drive'}, 
or \code{'user <profile value>'}.
\end{funcdesc}

\begin{funcdesc}{battery}{}
Returns the current battery level. On devices based on S60 2nd Edition
Feature Pack 1 (S60 2.1) or earlier the value ranges from 0 (empty) to
7 (full). On newer devices the value ranges from 0 (empty) to 100
(full). On the emulator the value is always 0. 
\begin{notice}[note] 
The returned value may be incorrect while the device is being charged.
\end{notice}

\end{funcdesc}


\begin{funcdesc}{display_twips}{}

Returns the width and height of the display in twips. For a definition of a 
twip, see Chapter \ref{sec:terms}, Terms and Abbreviations. 

\end{funcdesc}

\begin{funcdesc}{display_pixels}{}
Returns the width and height of the display in pixels.
\end{funcdesc}

\begin{funcdesc}{free_drivespace}{}
Returns the amount of free space left on the drives in bytes, for example 
\code{{\{}u'C:' 100{\}}.} The keys in the dictionary are the drive letters 
followed by a colon (:). 
\end{funcdesc}

\begin{funcdesc}{imei}{}
Returns the IMEI code of the device as a Unicode string or, if running
on the emulator, the hardcoded string \code{u'000000000000000'}.
\end{funcdesc}

\begin{funcdesc}{max_ramdrive_size}{}
Returns the maximum size of the RAM drive on the device.
\end{funcdesc}

\begin{funcdesc}{total_ram}{}
Returns the amount of RAM memory on the device.
\end{funcdesc}

\begin{funcdesc}{free_ram}{}
Returns the amount of free RAM memory available on the device.
\end{funcdesc}

\begin{funcdesc}{total_rom}{}
Returns the amount of read-only ROM memory on the device.
\end{funcdesc}

\begin{funcdesc}{ring_type}{}\textbf{Not supported in 1st Edition!}
Returns the current ringing type as a string, which can be one of the 
following: \code{'normal'}, \code{'ascending'}, 
\code{'ring_once'},\code{ 'beep'}, or \code{'silent'}.
\end{funcdesc}

\begin{funcdesc}{os_version}{}

Returns the operating system version number of the device as a three element tuple (major version, minor version, build number). The 
elements are as follows\footnote{ 
Descriptions for these values are based on information found in S60 SDK 
documentation \cite{S60Doc}.}:

\begin{itemize}
\item The major version number, ranging from 0 to 127 inclusive
\item The minor version number, ranging from 0 to 99 inclusive
\item The build number, ranging from 0 to 32767 inclusive.
\end{itemize}
\end{funcdesc}

\begin{funcdesc}{signal_bars}{}
Returns the current network signal strength ranging from 0 to 7, with 0 
meaning no signal and 7 meaning a strong signal. If using an emulator, value 
0 is always returned. 
\end{funcdesc}

\begin{funcdesc}{signal_dbm}{}
Returns the current network signal strength in dBm. This is available SDK 2.8
onwards. If using an emulator value 0 is always returned. 
\end{funcdesc}


\begin{funcdesc}{sw_version}{}
Returns the software version as a Unicode string. On the emulator,
returns the hardcoded string \textsf{u'emulator'}. For example, a
software version can be returned as \code{u'V 4.09.1 26-02-04 NHL-10 (c) NMP'}.
\end{funcdesc}
