% -*- tex -*-
% Copyright (c) 2005 - 2007 Nokia Corporation
%
% Licensed under the Apache License, Version 2.0 (the "License");
% you may not use this file except in compliance with the License.
% You may obtain a copy of the License at
%
%     http://www.apache.org/licenses/LICENSE-2.0
%
% Unless required by applicable law or agreed to in writing, software
% distributed under the License is distributed on an "AS IS" BASIS,
% WITHOUT WARRANTIES OR CONDITIONS OF ANY KIND, either express or implied.
% See the License for the specific language governing permissions and
% limitations under the License.

\chapter{Introduction}
\label{intro}

The Python for S60 Platform (Python for S60) simplifies application development 
and provides a scripting solution for the Symbian C++ APIs. This document is for 
Python for S60 release 1.4.5 release that is based on Python 2.2.2.

The documentation for Python for S60 includes three documents:

\begin{itemize}
\item Getting Started with Python for S60 Platform \cite{PyS60Start} contains information on how to install Python for S60 and how to write your first program.
\item This document contains API and other reference material.
\item Programming with Python for S60 Platform \cite{PyS60Prog} contains code examples and programming patterns for S60 devices that can be used as a basis for programs.
\end{itemize}
The Python for S60 as installed on a S60 device consists of:
\begin{itemize}
  \item Python runtime package that consists of:
  \begin{itemize}
   \item Python interpreter DLL
   \item Standard and proprietary Python library modules
   \item S60 UI application framework adaptation component (a DLL) that connects the scripting domain components to the S60 UI
  \end{itemize}
  \item Python script shell package that consists of:
  \begin{itemize}
   \item an application written in Python and visible in the application menu of the device that provides an execution environment for Python scripts.
   \item For S60 platform versions prior to 3rd Edition: Python Installer program for installing Python files on the device, which consists of:
 \begin{itemize} 
  \item A recognizer plug-in that recognizes .py, .pyc, .pyd and .pyo files as belonging to Python.
  \item Symbian	application written in Python that handles the installation of recognized Python files into the script shell environment.
 \end{itemize}
 \end{itemize}
\end{itemize}

A plugin for the S60 C++ SDK is also available. This plugin makes it
possible to run Python scripts in the S60 emulator environment and to
compile Python extension modules (PYDs) for the emulator and the
device.

The Python for S60 developer discussion board \cite{PyS60DiBo} on the 
Forum Nokia Web site is a useful resource for finding out information on 
specific topics concerning Python for S60. You are welcome to give 
feedback or ask questions about Python for S60 through this discussion 
board.

\section{Scope}
\label{subsec:scope}

This document includes the information required by developers to create 
applications that use Python for S60, and some advice on extending the 
platform.

\section{Audience}
\label{subsec:audience}

This guide is intended for developers looking to create programs that use the 
native features and resources of the S60 phones. The reader should be 
familiar with the Python programming language (\url{http://www.python.org/}) and 
the basics of using Python for S60 (see Getting Started with Python for 
S60 Platform \cite{PyS60Start}).


\section{Naming Conventions}
\label{subsec:naming}

Most names of the type \code{ESomething} typically indicate a constant defined 
by the Symbian SDK. More information about these constants can be found in the 
Symbian SDK documentation.
